\documentclass{article}

\usepackage{arxiv}

\usepackage{fontspec}
\setmainfont{Linux Libertine O}

\usepackage{polyglossia}
\setdefaultlanguage{english}
\setotherlanguage{russian}

\usepackage{url}
\usepackage{booktabs}
\usepackage{amsfonts}
\usepackage{amssymb}
\usepackage{amsmath}
\usepackage{nicefrac}
\usepackage{microtype}
\usepackage{lipsum}
\usepackage{graphicx}

\usepackage[numbers]{natbib}
\usepackage{amsthm}
\usepackage{doi}
\usepackage{xcolor}

\usepackage{todonotes}
\newcommand{\todoask}[1]{\todo[color=blue!40]{\textbf{Ask:} #1}}
\newcommand{\todocheck}[1]{\todo[color=green!40]{\textbf{Check:} #1}}
\newcommand{\todoread}[1]{\todo[color=orange!40]{\textbf{Read:} #1}}
\newcommand{\todocomment}[1]{\todo[color=purple!40]{\textbf{Comment:} #1}}


\title{Comparative Analysis of Data-Driven Approaches for Hydrological Forecasting}

\author{%
    Eldar Khuzin \\
    Chair of Data Analysis\\
    Moscow Institute of Physics and Technology\\
    Moscow, Russia \\
    \texttt{khuzin.er@phystech.su} \\
    \And
    Novikov Ivan \\
    Affiliation \\
    Address \\
    \texttt{email} \\
    \AND
    Abramov Dmitrii \\
    Affiliation \\
    Address \\
    \texttt{email}
}
\date{}

\hypersetup{
    pdftitle={Comparative Analysis of Data-Driven Approaches for Hydrological Forecasting},
    pdfsubject={cs.LG},
    pdfauthor={Eldar Khuzin, Novikov Ivan, Abramov Dmitrii},
    pdfkeywords={Flood discharge, Rainfall-runoff modeling, More}
}

\begin{document}
\maketitle

\begin{abstract}
	В данной работе проводится сравнительный анализ методов машинного обучения для решения задачи прогнозирования речного стока и уровня воды. Рассматриваются классические алгоритмы (Random Forest и CatBoost), отличающиеся сравнительно низкой вычислительной сложностью и быстрым обучением, рекуррентные нейронные сети с долговременной памятью (LSTM), способные учитывать долгосрочные зависимости в данных, и современные архитектуры на основе трансформеров, которые демонстрируют перспективные результаты. Эксперименты проводятся на гидрологических данных рек России и Франции, содержащие динамические временные ряды гидрологических измерений и статические характеристики водосборных бассейнов. Цель работы — выявить наиболее эффективные подходы к прогнозировани.
\end{abstract}

\keywords{Flood discharge \and Rainfall-runoff modeling \and More}

\section{Introduction}
\label{sec:introduction}

Прогнозирование речного стока и уровня воды является важной задачей для предупреждения наводнений и эффективного использования водных ресурсов в таких областях как сельское хозяйство и гидроэнергетика.\cite{laversVisionHydrologicalPrediction2020}

Современные подходы к моделированию гидрологических процессов можно разделить на два основных типа: физически обоснованные (physical-based) и основанные на данных (data-driven). Физические модели, несмотря на свою интерпретируемость и потенциальную точность, имеют ряд существенных недостатков: для их использования требуются обширные данные о речном бассейне и значительные вычислительные ресурсы, что ограничивает их практическое применение в задачах краткосрочного прогнозирования \cite{bellosHybridMethodFlood2016}. Также построение таких моделей требует глубокого понимания гидрологических процессов и сталкивается с проблемами обобщаемости.

Data-driven подходы привлекают отсутствием явного моделирования физических процессов. Классические алгоритмы машинного обучения, такие как Random Forest, показывают себя конкуретно способными в определённых условиях, выделяясь при этом быстрой обучаемостью и небольшими требованиями к вычислительным ресурсам\cite{schoppaEvaluatingPerformanceRandom2020}. Рекуррентные нейронные сети (Recurrent Neural Networks, RNN), в частности, Long Short-Term Memory (LSTM), также показывают потенциал за счет способности учитывать долгосрочные зависимости, такие как таянье снежного покрова или наличие накопленной влаги в почве\cite{kratzertRainfallRunoffModelling2018, kratzertHESSOpinionsNever2024}.Недавние исследования демонстрируют, что и трансформеры могут успешно применяться для гидрологических прогнозов\cite{liInterpretableHybridDeep2024,liuProbingLimitHydrologic2023}. Однако сравнительный анализ на разных типах гидрологических данных и в различных географических условиях представлен недостаточно.

В данной работе авторы ставят перед собой следующие цели: провести сравнительный анализ data-driven подходов, в частности с желанием побить метрики LSTM, а также исследовать влияние физико-географических атрибутов на результат. Оценка производится через такие метрики, как среднеквадратичная ошибка (RMSE) и коэффициент эффективности Нэша-Сатклиффа (Nash–Sutcliffe Efficiency, NSE).

\section{Problem statement}
\label{sec:problem_statement}

Используется метрика среднеквадратичной ошибки (MSE). А именно:
\[
	\operatorname{MSE}(f) = \frac{1}{N} \sum_{i=1}^N \left( y_i - f(x_i) \right)^2,
\]
где \( y_i \) --- истинное значение наблюдения, а \( f(x_i) \) --- предсказанное значение.

И ставится множество оптимизационных задач по нахождению оптимальной модели \( f^* \) в своём классе моделей \( \mathcal{F} \), т.е.:
\[
	f^* = \arg\min_{f \in \mathcal{F}} \operatorname{MSE}(f).
\]

Основное множество данных состоит из динамических временных рядов гидрологических измерений (уровня воды, речного стока, а также температуры) и статических характеристик водосборных бассейнов.

\todoask{Требуется ли описать каждую из введённых моделей}
\todocomment{Поговорить насчёт того, как каждую из моделей адаптировать под данную задачу.}

\bibliographystyle{unsrtnat}
\bibliography{references}

\end{document}
